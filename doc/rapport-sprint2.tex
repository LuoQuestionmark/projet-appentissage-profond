\documentclass{article}
\usepackage[utf8]{inputenc}
\usepackage{amssymb, amsmath}
\usepackage{parskip}
\usepackage{url}
\usepackage{listings}
\usepackage{xcolor}

\definecolor{codegreen}{rgb}{0,0.6,0}
\definecolor{codegray}{rgb}{0.5,0.5,0.5}
\definecolor{codepurple}{rgb}{0.58,0,0.82}
\definecolor{backcolour}{rgb}{0.95,0.95,0.92}

\lstdefinestyle{mystyle}{
    backgroundcolor=\color{backcolour},   
    commentstyle=\color{codegreen},
    keywordstyle=\color{magenta},
    numberstyle=\tiny\color{codegray},
    stringstyle=\color{codepurple},
    basicstyle=\ttfamily\footnotesize,
    breakatwhitespace=false,         
    breaklines=true,                 
    captionpos=b,                    
    keepspaces=true,                 
    numbers=left,                    
    numbersep=5pt,                  
    showspaces=false,                
    showstringspaces=false,
    showtabs=false,                  
    tabsize=2
}

\lstset{style=mystyle}

\title{Rapport (sprint 2) du projet de 8INF892}
\author{Wenhao LUO\\LUOW09129805}
\date{\today}

\begin{document}
\maketitle

\newpage
\section{Introduction}

\par Dans le cadre du cours 8INF892, on a besoin de réaliser un projet en utilisant les outils de l'apprentissage profond. On choisit de créer un réseau neurone qui permet de produire de la musique en imitant le style de Bach. Le projet sera réaliser en utiliisant \texttt{Python}, avec l'aide de la bibliothèque \texttt{TensorFlow}.

\par Pour le deuxième sprint, on a progressé en effectuant plusieurs tâches. Un bilan des tâches réalisées est présenté dans le tableau \ref{tab:bilan}. Dans les prochaines sections, on va présenter en détails des tâches réalisées.

\par Le dépôt du projet est disponible sur Github :\newline \url{https://github.com/LuoQuestionmark/projet-appentissage-profond}.

\begin{table}[htb]
    \centering
    \begin{tabular}{|l|c|}
        \hline
        nom de tâche & état \\
        \hline
        \hline
        préparation de données (premier modèle) & terminé \\
        \hline
        création du réseau (premier modèle) & terminé \\
        \hline
        préparation de données (deuxième modèle) & en train de faire \\
        \hline
        création du réseau (deuxième modèle) & à faire \\
        \hline
        mélange des résultats & à faire \\
        \hline
        présentation des résultats & à faire \\
        \hline
    \end{tabular}
    \caption{Le bilan sur les tâches réalisées ainsi que le plan pour le suivant.}
    \label{tab:bilan}
\end{table}

\section{Premier modèle}

\subsection{Préparation des données}

\par Dans le rapport du sprint I, nous avons mentionné l'utilisation de musiques de Bach au format \texttt{midi} comme données d'entraînement. À présent, nous avons divisé ces tâches en plusieurs étapes. Un premier programme a été développé pour lire et analyser les fichiers ; un deuxième programme permet de traduire ces données sous les format ``one hot", les données préparées sont ensuite utilsées pour l'entraînement du modèle.

\par L'ensemble des notes est enregistré par leur temps de début $t_s$, leur temps de fin $t_e$, et leur hauteur tonale $p$. On peut définir un espace $\mathbb{E} = \mathbb{N} \times \mathbb{N}$, une injection $f$, et toutes les notes peuvent être projectées cet espace par leur temps de début et leur hauteur tonale. On peut donc définir la notion de distance sur cet espace $\mathbb{E}$. La distance entre deux notes $n_1, n_2$ est définie par $d = ||n_1 - n_2||_2$. Cette notion permet de déduire des "voisins" d'une note donnée.

\subsection{Entraînement du modèle}

\par Si une note $n_1$ possède une liste de voisins $(n_i)_{i \in [2, n]}$, l'ensemble des notes peut être considéré comme une entrée du modèle. La liste des voisins est la variable, et la note $n_1$ est le label. Le but de ce premier modèle est de prédire une note manquante en utilisant tous ses voisins. À partir de ce modèle, le système peut imiter le style du compositeur.

\section{Deuxième modèle}

\par Le problème avec le premier modèle est le manque de connaissances spatiales. Actuellement, nous sommes en train de préparer un deuxième modèle qui permet d'encoder ces connaissances. Nous avons trouvé un article qui utilise des réseaux neuronaux RNN et LSTM pour effectuer une recherche similaire\cite{music-gen-RNN-LSTM}.

\section{Présentation de la sortie des réseaux}

\par En utilisant le premier modèle, nous sommes déjà capables de créer des notes musicales à partir de quelques notes données. Pour l'instant la sortie n'est pas vraiment lisible, un des exemples est donné dans le listing \ref{lst:first-mod-out}. Chaque fois le système essaie de prédire une note, et le résultat est marqué comme ``drop".

\begin{lstlisting}[caption={Une sortie du premier modèle.}, label={lst:first-mod-out}]
    1/1 [==============================] - 1s 1s/step
    drop: 55
    1/1 [==============================] - 0s 15ms/step
    drop: 60
    1/1 [==============================] - 0s 17ms/step
    drop: 62
    1/1 [==============================] - 0s 16ms/step
    drop: 64
    1/1 [==============================] - 0s 14ms/step
    drop: 59
    1/1 [==============================] - 0s 14ms/step
    drop: 67
    1/1 [==============================] - 0s 14ms/step
    drop: 55
    1/1 [==============================] - 0s 16ms/step
    drop: 62
    1/1 [==============================] - 0s 15ms/step
    drop: 64
    1/1 [==============================] - 0s 17ms/step
    drop: 65
    1/1 [==============================] - 0s 15ms/step
\end{lstlisting}

\par Ce n'est donc pas un format lisible pour humain ; pour le rendu final du projet, on planifie à utiliser le format du \texttt{Lilypond} comme la sortie. C'est un logiciel du projet GNU\footnote{\url{http://lilypond.org/}}.

\section{Conclusion}

\par Dans le deuxième sprint, on a réalisé un premier modèle et on prévoit d'utiliser un deuxième modèle pour la suite. On a aussi traité des données et on a retrouvé un format pour la sortie de notre programme. 

\bibliographystyle{plain}
\bibliography{ref}

\end{document}
